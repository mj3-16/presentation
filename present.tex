\documentclass{beamer}
\usetheme[deutsch]{KIT}

\usepackage[utf8]{inputenc}
\usepackage[T1]{fontenc}
\usepackage{babel}
\usepackage{tikz,calc,ifthen}
\usepackage{mathtools}
\usepackage[normalem]{ulem}
\usepackage{graphicx}
\usepackage{minted}
\usepackage[]{algorithm2e}
\usepackage{pgfplots,pgfplotstable}
\usepgfplotslibrary{dateplot}
\usepackage{filecontents}

\usetikzlibrary{positioning,calc,arrows,shapes}
\tikzset{
  every node/.style={transform shape},
  auto,
  block/.style={align=center,rectangle,draw,minimum height=20pt,minimum width=30pt},
  >=triangle 60,
  alt/.code args={<#1>#2#3}{%
      \alt<#1>{\pgfkeysalso{#2}}{\pgfkeysalso{#3}}
  },
  beameralert/.style={alt=<#1>{color=green!80!black}{}},
  mythick/.style={line width=1.4pt}
}

\newcommand*{\maxwidthofm}[2]{\maxof{\widthof{$#1$}}{\widthof{$#2$}}}
\newcommand<>*{\robustaltm}[2]{
  \alt#3
  {\mathmakebox[\maxwidthofm{#1}{#2}]{#1}\vphantom{#1#2}}
    {\mathmakebox[\maxwidthofm{#1}{#2}]{#2}\vphantom{#1#2}}
}

\newcommand<>*{\nodealert}[1]{\only#2{\draw[overlay,mythick,color=green!80!black] (#1.north west) rectangle (#1.south east)}}

\title{Optimierungsphase und Tests}
\author{Gruppe 3}
\subtitle{\insertauthor}
\institute[IPD]{Lehrstuhl Programmierparadigmen, IPD Snelting}
\date{29.10.2013}
\KITtitleimage{images/cover}
\setlength{\titleimageht}{100cm}

\newenvironment{javacode}{\begin{minted}[escapeinside=||]{java}}{\end{minted}}

\newcommand\Wider[2][3em]{%
	\makebox[\linewidth][c]{%
		\begin{minipage}{\dimexpr\textwidth+#1\relax}
			\raggedright#2
		\end{minipage}%
	}%
}

\begin{document}

\begin{frame}
  \maketitle
\end{frame}

\begin{frame}[fragile]{Ziel von Optimierungen}
	\begin{minted}[escapeinside=||]{java}
class Test {

	public boolean run(){
		int size = 10;
		int[] result = createArray(size, size * 3 + 4);
		return result[0] == size * 3 + 4;
	}
		
	public int[] createArray(int size, int value){
		int[] arr = new int[size];
		arr[0] = value;
		return arr;
	}
}
	\end{minted}
\end{frame}

\begin{frame}[fragile]{Ziel von Optimierungen}
	\begin{minted}[escapeinside=||]{java}
class Test {

	public boolean run(){
	
	
		return true;
	}
	
	public int[] createArray(int size, int value){
		int[] arr = new int[size];
		arr[0] = value;
		return arr;
	}
}
	\end{minted}
\end{frame}

\begin{frame}{Verband}
	\textit{Grafik mti Verband, unten "`Unwissen"', oben "`zu viel Wissen"'}
\end{frame}

\begin{frame}{Optimierungsalgorithmus}
	\begin{algorithm}[H]
		\KwData{Firm graph}
		\KwResult{Optimized firm graph}
		\KwData{Information for each node}
		\While{Information for each node changed}{
			Walk the graph topological and update information\;.
		}
		Simplification\;
	\end{algorithm}
\end{frame}

\begin{frame}[fragile]{Beispiel im Folgenden}
\begin{minted}[escapeinside=||]{java}
class Test {

	public boolean run(){
		int size = 10;
		int[] result = createArray(size, size * 3 + 4);
		return result[0] == size * 3 + 4;
	}
	
	public int[] createArray(int size, int value){
		int[] arr = new int[size];
		arr[0] = value;
		return arr;
	}
}
\end{minted}
\end{frame}

\newcommand\myonly[2]{\only<#1>{#2}}
\newcommand{\highlight}{\color{orange}}
\newcommand{\myhonly}[2]{{\highlight\myonly{#1}{#2}}}
\newcommand{\mymonly}[2]{{\myonly{#1}{#2}}}

\begin{frame}[fragile]{Common Subexpression Elemination}
\begin{minted}[escapeinside=||]{java}
class Test {
	
	public boolean run(){
		int size = 10;
		|\myhonly{2-}{int imm = size * 3 + 4;}|
		int[] result = createArray(size, |\myhonly{1}{size * 3 + 4}\myhonly{2-}{imm}|);
		return result[0] == |\myhonly{1}{size * 3 + 4}\myonly{2-}{imm}|);
	}
	
	public int[] createArray(int size, int value){
		int[] arr = new int[size];
		arr[0] = value;
		return arr;
	}
}
\end{minted}
\end{frame}

\begin{frame}[fragile]{Constant Propagation}
\begin{minted}[escapeinside=||]{java}
class Test {

	public boolean run(){
		|\myhonly{1}{int size = 10;}|
		|\mymonly{-4}{int imm =}\myhonly{1}{ size}\myhonly{2}{ 10}\myonly{1-2}{ * 3 + 4}\myhonly{3}{ 30 + 4}\myhonly{4}{ 34}\mymonly{-4}{;}\myhonly{5}{int imm = 34;}|
		int[] result = createArray(|\myhonly{1}{size}\myhonly{2}{10}\mymonly{3-}{10}|, |\mymonly{-4}{imm}\myhonly{5}{imm}\myhonly{6}{34}|);
		return result[0] == imm;
	}

	public int[] createArray(int size, int value){
		int[] arr = new int[size];
		arr[0] = value;
		return arr;
	}
}
\end{minted}
\end{frame}


\begin{frame}[fragile]{Inlining}
	\begin{minted}[escapeinside=||]{java}
class Test {
	
	public boolean run(){
		int[] result = |\myhonly{1}{createArray(10, 34)}|;
		return result[0] == 34;
	}
	
	public int[] createArray(int size, int value){
		int[] arr = new int[size];
		arr[0] = value;
		return arr;
	}
}
	\end{minted}
\end{frame}

\begin{frame}[fragile]{Inlining}
\begin{minted}[escapeinside=||]{java}
class Test {

	public boolean run(){
		int value = 34;
		int[] arr = new int[size];
		arr[0] = value;
		int[] result = arr;
		return result[0] == 34;
	}
	
	public int[] createArray(int size, int value){
		int[] arr = new int[size];
		arr[0] = value;
		return arr;
	}
}
\end{minted}
\end{frame}

\begin{frame}[fragile]{Cleaning up after inlining}
	\begin{minted}[escapeinside=||]{java}
class Test {
	
	public boolean run(){
		int[] arr = new int[size];
		arr[0] = 34;
		return arr[0] == 34;
	}
	
	public int[] createArray(int size, int value){
		int[] arr = new int[size];
		arr[0] = value;
		return arr;
	}
}
\end{minted}
\end{frame}

\begin{frame}[fragile]{Alias Analysis}
	\begin{minted}[escapeinside=||]{java}
class Test {
	
	public boolean run(){
		|\myonly{1}{int[] arr = new int[size];}|
		|\myhonly{1}{arr[0] = 34;}|
		return |\myhonly{1}{arr[0]}\myhonly{2}{34}| == 34;
	}
	
	public int[] createArray(int size, int value){
		int[] arr = new int[size];
		arr[0] = value;
		return arr;
	}
}
	\end{minted}
\end{frame}

\begin{frame}[fragile]{Constant Propagation}
\begin{minted}[escapeinside=||]{java}
class Test {
	
	public boolean run(){
		return |\myhonly{1}{34 == 34}\myhonly{2}{true}|;
	}
	
	public int[] createArray(int size, int value){
		int[] arr = new int[size];
		arr[0] = value;
		return arr;
	}
}
\end{minted}
\end{frame}

\begin{frame}{Fixpunktiteration}
	\begin{algorithm}[H]
		\KwData{Firm graph}
		\KwResult{Optimized firm graph}
		\KwData{Information for each node}
		\While{Graph changed}{
			\ForEach{Optimization}{Use it}
		}
	\end{algorithm}
\end{frame}

\begin{frame}[fragile]{mjTest}
	\begin{minted}[escapeinside=||]{bash}
mjt.py [-h] [--only_incorrect_tests] 
	[--parallel]
	[--ci_testing]
	{all,
	lexer,
	syntax,
	ast,
	semantic,
	compile-firm,
	,compile}
	MJ_RUN
	\end{minted}
\end{frame}

	\begin{filecontents*}{exec.csv}
	d, c
	16-11-17, 10
	16-11-22, 3
	16-11-28, 72
	16-12-10, 1
	16-12-28, 2
	17-01-04, 4
	17-01-15, 6
	17-01-16, 4
	17-01-23, 2
\end{filecontents*}


\begin{filecontents*}{semantic.csv}
	d, c
	16-11-09, 56
	16-11-18, 157
	16-11-22, 17
	16-12-03, 2
	16-12-27, 6
\end{filecontents*}


\begin{filecontents*}{syntax.csv}
	d, c
	16-10-29, 1
	16-11-04, 80
	16-11-09, 12
	16-11-16, 4
	16-12-27, 3
\end{filecontents*}


\begin{filecontents*}{lexer.csv}
	d, c
	16-11-07, 7
	16-11-16, 1
\end{filecontents*}

\begin{filecontents*}{semantic2.csv}
	d, c
	16-11-09, 6266
	16-11-18, 19205
	16-11-22, 1730
	16-12-03, 211
	16-12-27, 635
\end{filecontents*}


\begin{filecontents*}{exec2.csv}
	d, c
	16-11-17, 3627
	16-11-22, 7431
	16-11-28, 31106
	16-12-10, 251
	16-12-28, 318
	17-01-04, 735
	17-01-15, 1443
	17-01-16, 744
	17-01-23, 340
\end{filecontents*}


\begin{filecontents*}{syntax2.csv}
	d, c
	16-10-29, 32
	16-11-04, 8177
	16-11-09, 1088
	16-11-16, 303
	16-12-27, 294
\end{filecontents*}


\begin{filecontents*}{lexer2.csv}
	d, c
	16-11-07, 673
	16-11-16, 25
\end{filecontents*}


\begin{frame}{mjTest \textendash Statistiken}
	\only<1>{
		\centering
		    \begin{tikzpicture}
	\begin{axis}[
	symbolic x coords={1, 2, 3, 4, 5},
	xtick=data,
	ylabel={Test files},
	xlabel={Gruppe}
	]
	\addplot[ybar,fill=blue] coordinates {
		(1, 33)
		(2, 28)
		(3, 259)
		(4, 102)
		(5, 28)
	};
	\end{axis}
	\end{tikzpicture}	
}
\only<2>{

	\begin{tikzpicture}
\begin{axis}[
width=10cm,
height=6cm,
date coordinates in=x,
xlabel=Month,
xticklabel={\pgfcalendarmonthshortname{\month}}, % short name of month for ticklabels
xtick={ % ticks at first day of each month
	16-11-01,
	16-12-01,
	16-10-01,
	17-01-01,
	17-02-01},
xticklabel style={font=\footnotesize},
ylabel={Test file count},
ymin=0
]

\addplot table [x=d,y=c, col sep=comma] {lexer.csv};
\addplot table [x=d,y=c, col sep=comma] {syntax.csv};
\addplot table [x=d,y=c, col sep=comma] {semantic.csv};
\addplot table [x=d,y=c, col sep=comma] {exec.csv};

\addlegendentry{Lexer}
\addlegendentry{Parser}
\addlegendentry{Semantic}
\addlegendentry{Execute}

\end{axis}
\end{tikzpicture}
}
\only<3>{
\begin{tikzpicture}
\begin{axis}[
width=10cm,
height=6cm,
date coordinates in=x,
xlabel=Month,
xticklabel={\pgfcalendarmonthshortname{\month}}, % short name of month for ticklabels
xtick={ % ticks at first day of each month
	16-11-01,
	16-12-01,
	16-10-01,
	17-01-01,
	17-02-01},
xticklabel style={font=\footnotesize},
ylabel={Test files gzipped},
ymin=0
]

\addplot table [x=d,y=c, col sep=comma] {lexer2.csv};
\addplot table [x=d,y=c, col sep=comma] {syntax2.csv};
\addplot table [x=d,y=c, col sep=comma] {semantic2.csv};
\addplot table [x=d,y=c, col sep=comma] {exec2.csv};

\addlegendentry{Lexer}
\addlegendentry{Parser}
\addlegendentry{Semantic}
\addlegendentry{Execute}

\end{axis}
\end{tikzpicture}
}
\end{frame}

\begin{frame}{Zusammenfassung}
\begin{itemize}
	\item Viel Arbeit und viel Spaß
	\item n Zeilen Java Code
	\item \dots
\end{itemize}
\end{frame}

\end{document}
