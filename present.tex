\documentclass{beamer}
\usetheme[deutsch]{KIT}

\usepackage[utf8]{inputenc}
\usepackage[T1]{fontenc}
\usepackage{babel}
\usepackage{tikz,calc,ifthen}
\usepackage{mathtools}
\usepackage[normalem]{ulem}
\usepackage{graphicx}
\usepackage{minted}
\usepackage[]{algorithm2e}
\usepackage{pgfplots,pgfplotstable}
\usepgfplotslibrary{dateplot}
\usepackage{filecontents}

\usetikzlibrary{positioning,calc,arrows,shapes}
\tikzset{
  every node/.style={transform shape},
  auto,
  block/.style={align=center,rectangle,draw,minimum height=20pt,minimum width=30pt},
  >=triangle 60,
  alt/.code args={<#1>#2#3}{%
      \alt<#1>{\pgfkeysalso{#2}}{\pgfkeysalso{#3}}
  },
  beameralert/.style={alt=<#1>{color=green!80!black}{}},
  mythick/.style={line width=1.4pt}
}

\newcommand*{\maxwidthofm}[2]{\maxof{\widthof{$#1$}}{\widthof{$#2$}}}
\newcommand<>*{\robustaltm}[2]{
  \alt#3
  {\mathmakebox[\maxwidthofm{#1}{#2}]{#1}\vphantom{#1#2}}
    {\mathmakebox[\maxwidthofm{#1}{#2}]{#2}\vphantom{#1#2}}
}

\newcommand<>*{\nodealert}[1]{\only#2{\draw[overlay,mythick,color=green!80!black] (#1.north west) rectangle (#1.south east)}}

\title{Optimierungsphase und Tests}
\author{Gruppe 3}
\subtitle{\insertauthor}
\institute[IPD]{Lehrstuhl Programmierparadigmen, IPD Snelting}
\date{29.10.2013}
\KITtitleimage{images/cover}
\setlength{\titleimageht}{100cm}

\newenvironment{javacode}{\begin{minted}[escapeinside=||]{java}}{\end{minted}}

\newcommand\Wider[2][3em]{%
	\makebox[\linewidth][c]{%
		\begin{minipage}{\dimexpr\textwidth+#1\relax}
			\raggedright#2
		\end{minipage}%
	}%
}

\begin{document}

\begin{frame}
  \maketitle
\end{frame}

\begin{frame}[fragile]{Ziel von Optimierungen}
	\begin{minted}[escapeinside=||]{java}
class Test {

	public boolean run(){
		int size = 10;
		int[] result = createArray(size, size * 3 + 4);
		return result[0] == size * 3 + 4;
	}
		
	public int[] createArray(int size, int value){
		int[] arr = new int[size];
		arr[0] = value;
		return arr;
	}
}
	\end{minted}
\end{frame}

\begin{frame}[fragile]{Ziel von Optimierungen}
	\begin{minted}[escapeinside=||]{java}
class Test {

	public boolean run(){
	
	
		return true;
	}
	
	public int[] createArray(int size, int value){
		int[] arr = new int[size];
		arr[0] = value;
		return arr;
	}
}
	\end{minted}
\end{frame}

\begin{frame}{Verband}
	\textit{Grafik mit Verband, unten "`Unwissen"', oben "`zu viel Wissen"'}
\end{frame}

\begin{frame}[fragile]{Beispiel im Folgenden}
\begin{minted}[escapeinside=||]{java}
class Test {

	public boolean run(){
		int size = 10;
		int[] result = createArray(size, size * 3 + 4);
		return result[0] == size * 3 + 4;
	}
	
	public int[] createArray(int size, int value){
		int[] arr = new int[size];
		arr[0] = value;
		return arr;
	}
}
\end{minted}
\end{frame}

\newcommand\myonly[2]{\only<#1>{#2}}
\newcommand{\highlight}{\color{orange}}
\newcommand{\myhonly}[2]{{\highlight\myonly{#1}{#2}}}
\newcommand{\mymonly}[2]{{\myonly{#1}{#2}}}

\begin{frame}[fragile]{Common Subexpression Elemination}
%\only<1>{\setminted{highlightlines={6,7}}}
%\only<2>{\setminted{highlightlines={5-7}}}
\begin{minted}[escapeinside=||]{java}
class Test {
	
	public boolean run(){
		int size = 10;
		|\myhonly{2-}{int imm = size * 3 + 4;}|
		int[] result = createArray(size, |\myhonly{1}{size * 3 + 4}\myhonly{2-}{imm}|);
		return result[0] == |\myhonly{1}{size * 3 + 4}\myhonly{2-}{imm}|;
	}
	
	public int[] createArray(int size, int value){
		int[] arr = new int[size];
		arr[0] = value;
		return arr;
	}
}
\end{minted}
\end{frame}

\begin{frame}[fragile]{Constant Propagation}
\begin{minted}[escapeinside=||]{java}
class Test {

	public boolean run(){
		|\myhonly{1-2}{int size = 10;}|
		|\mymonly{-6}{int imm =}\mymonly{1}{ size * 3 + 4}\myhonly{2}{ size}\myhonly{3}{ 10}\myonly{2-3}{ * 3 + 4}\myhonly{4}{ 10 * 3}\mymonly{4}{ + 4}\myhonly{5}{ 30 + 4}\myhonly{6}{ 34}\mymonly{-6}{;}\myhonly{7}{int imm = 34;}|
		int[] result = createArray(|\myonly{1}{size}\myhonly{2}{size}\myhonly{3}{10}\mymonly{4-}{10}|, |\mymonly{-6}{imm}\myhonly{7}{imm}\myhonly{8}{34}|);
		return result[0] == |\mymonly{-6}{imm}\myhonly{7}{imm}\myhonly{8}{34}|;
	}

	public int[] createArray(int size, int value){
		int[] arr = new int[size];
		arr[0] = value;
		return arr;
	}
}
\end{minted}
\end{frame}


\begin{frame}[fragile]{Inlining}
	\begin{minted}[escapeinside=||]{java}
class Test {
	
	public boolean run(){
		int[] result = |\myhonly{1}{createArray(10, 34)}|;
		return result[0] == 34;
	}
	
	public int[] createArray(int size, int value){
		int[] arr = new int[size];
		arr[0] = value;
		return arr;
	}
}
	\end{minted}
\end{frame}

\begin{frame}[fragile]{Inlining}
\begin{minted}[escapeinside=||, highlightlines={4-7}]{java}
class Test {

	public boolean run(){
		int value = 34;
		int[] arr = new int[size];
		arr[0] = value;
		int[] result = arr;
		return result[0] == 34;
	}
	
	public int[] createArray(int size, int value){
		int[] arr = new int[size];
		arr[0] = value;
		return arr;
	}
}
\end{minted}
\end{frame}

\begin{frame}[fragile]{Cleaning up after inlining}
	\begin{minted}[escapeinside=||, highlightlines={4-5}]{java}
class Test {
	
	public boolean run(){
		int[] arr = new int[size];
		arr[0] = 34;
		return arr[0] == 34;
	}
	
	public int[] createArray(int size, int value){
		int[] arr = new int[size];
		arr[0] = value;
		return arr;
	}
}
\end{minted}
\end{frame}

\begin{frame}[fragile]{Alias Analysis}
	\begin{minted}[escapeinside=||]{java}
class Test {
	
	public boolean run(){
		|\myonly{1}{int[] arr = new int[size];}|
		|\myhonly{1}{arr[0]}\mymonly{1}{ = 34;}|
		return |\myhonly{1}{arr[0]}\myhonly{2}{34}| == 34;
	}
	
	public int[] createArray(int size, int value){
		int[] arr = new int[size];
		arr[0] = value;
		return arr;
	}
}
	\end{minted}
\end{frame}

\begin{frame}[fragile]{Constant Propagation}
\begin{minted}[escapeinside=||]{java}
class Test {
	
	public boolean run(){
		return |\myhonly{1}{34 == 34}\myhonly{2}{true}|;
	}
	
	public int[] createArray(int size, int value){
		int[] arr = new int[size];
		arr[0] = value;
		return arr;
	}
}
\end{minted}
\end{frame}

\begin{frame}{Fix Point Iteration}
	\centering
	\begin{algorithm}[H]
		\KwData{Firm graph}
		\KwResult{Optimized firm graph}
		\KwData{Information for each node}
		\While{Graph changed}{
			\ForEach{Optimization}{Use it}
		}
	\end{algorithm}
\end{frame}

\begin{frame}{mjtest}
	\begin{itemize}
		\item Compiler testing framework
		\item \dots and the curated test cases of all groups
		\item 5 test categories:
		\begin{itemize}
			\item \textit{lexer}
			\item \textit{syntax}
			\item \textit{ast}
			\item \textit{semantic}
			\item \textit{exec}
		\end{itemize}
		\item Usable for ci-testing
	\end{itemize}
\end{frame}
           \begin{filecontents*}{semantic.csv}
	d, c
	16-10-29, 1
	16-11-02, 14
	16-11-03, 54
	16-11-04, 81
	16-11-07, 88
	16-11-09, 156
	16-11-14, 169
	16-11-15, 209
	16-11-16, 224
	16-11-17, 256
	16-11-18, 272
	16-11-19, 296
	16-11-20, 318
	16-11-21, 323
	16-11-22, 335
	16-11-28, 335
	16-11-29, 336
	16-11-30, 336
	16-12-01, 336
	16-12-02, 336
	16-12-03, 337
	16-12-10, 337
	16-12-27, 346
	16-12-28, 346
	16-12-30, 346
	17-01-04, 346
	17-01-11, 346
	17-01-13, 346
	17-01-15, 346
	17-01-16, 346
	17-01-17, 346
	17-01-23, 346
\end{filecontents*}


\begin{filecontents*}{syntax.csv}
	d, c
	16-10-29, 1
	16-11-02, 14
	16-11-03, 54
	16-11-04, 81
	16-11-07, 88
	16-11-09, 100
	16-11-14, 100
	16-11-15, 100
	16-11-16, 105
	16-11-17, 105
	16-11-18, 105
	16-11-19, 105
	16-11-20, 105
	16-11-21, 105
	16-11-22, 105
	16-11-28, 105
	16-11-29, 105
	16-11-30, 105
	16-12-01, 105
	16-12-02, 105
	16-12-03, 105
	16-12-10, 105
	16-12-27, 108
	16-12-28, 108
	16-12-30, 108
	17-01-04, 108
	17-01-11, 108
	17-01-13, 108
	17-01-15, 108
	17-01-16, 108
	17-01-17, 108
	17-01-23, 108
\end{filecontents*}


\begin{filecontents*}{exec.csv}
	d, c
	16-10-29, 1
	16-11-02, 14
	16-11-03, 54
	16-11-04, 81
	16-11-07, 88
	16-11-09, 156
	16-11-14, 169
	16-11-15, 209
	16-11-16, 224
	16-11-17, 264
	16-11-18, 280
	16-11-19, 304
	16-11-20, 328
	16-11-21, 333
	16-11-22, 348
	16-11-28, 350
	16-11-29, 375
	16-11-30, 404
	16-12-01, 412
	16-12-02, 421
	16-12-03, 422
	16-12-10, 423
	16-12-27, 432
	16-12-28, 433
	16-12-30, 434
	17-01-04, 438
	17-01-11, 439
	17-01-13, 440
	17-01-15, 444
	17-01-16, 445
	17-01-17, 448
	17-01-23, 450
\end{filecontents*}


\begin{filecontents*}{lexer.csv}
	d, c
	16-10-29, 0
	16-11-02, 0
	16-11-03, 0
	16-11-04, 0
	16-11-07, 7
	16-11-09, 7
	16-11-14, 7
	16-11-15, 7
	16-11-16, 8
	16-11-17, 8
	16-11-18, 8
	16-11-19, 8
	16-11-20, 8
	16-11-21, 8
	16-11-22, 8
	16-11-28, 8
	16-11-29, 8
	16-11-30, 8
	16-12-01, 8
	16-12-02, 8
	16-12-03, 8
	16-12-10, 8
	16-12-27, 8
	16-12-28, 8
	16-12-30, 8
	17-01-04, 8
	17-01-11, 8
	17-01-13, 8
	17-01-15, 8
	17-01-16, 8
	17-01-17, 8
	17-01-23, 8
\end{filecontents*}

\begin{frame}{Test files over time}
	\centering

	\begin{tikzpicture}
\begin{axis}[
width=12cm,
height=7.6cm,
ymax=700,
date coordinates in=x,
grid=both,
major grid style={line width=.1pt, draw=gray!10},
%xlabel=Month,
%xticklabel={\pgfcalendarmonthshortname{\month}},
%xtick={ % ticks at first day of each month
%	16-11-01,
%	16-12-01,
%	16-10-01,
%	17-01-01,
%	17-02-01},
xticklabels={Lexer,Parser,AST,Semantic,IR,Optimization,Backend,2017},
xlabel style={yshift=-1cm},
x tick label style={
	rotate=45,
	anchor=east,
},
xtick={16-10-17, 16-10-24, 16-11-07, 16-11-14, 16-11-21, 16-12-05, 16-12-12, 17-01-01},
xticklabel style={font=\footnotesize},
%ylabel={Test file count},
ymin=0
]

\addplot table [x=d,y=c, mark=, col sep=comma] {lexer.csv};
\addplot table [x=d,y=c, mark=, col sep=comma] {syntax.csv};
\addplot table [x=d,y=c, mark=, col sep=comma] {semantic.csv};
\addplot table [x=d,y=c, mark=, col sep=comma] {exec.csv};

\addlegendentry[mark=]{Lexer}
\addlegendentry{Parser}
\addlegendentry{Semantic}
\addlegendentry{Execute}

\end{axis}
\end{tikzpicture}

\end{frame}

\begin{frame}{Interesting \textit{exec} Tests}
\begin{itemize}
	\item Conways game of life
	\item Math expression interpreter
	\item Mersenne twister random number generator
	\item Quine
	\item Turing machine
	\item \textit{binarytrees} and \textit{fannkuchredux} of the \textit{Benchmarks game}
\end{itemize}
\end{frame}

\begin{frame}[fragile]{Preprocessor}
\begin{minted}[escapeinside=||]{java}
import lib.ArrayList;

class Test {
	ArrayList<Integer> elements;
	int[] arr;
	
	public Test(int size){
		elements = new ArrayList<Integer>(size);
		arr = new int[]{size, size + 1};
	}
}
\end{minted}
\end{frame}

\begin{frame}{Summary}
\begin{itemize}
	\item Viel Arbeit und viel Spaß
	\item n Zeilen Java Code
	\item \dots
\end{itemize}
\end{frame}

\end{document}
